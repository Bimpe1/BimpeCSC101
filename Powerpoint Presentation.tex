\documentclass{article}
\begin{document}
   \textbf{\underline{Historical Perspective On Some Programming Languages}}
  \section*\underline\textbf{What Are Programming Languages?}
    A programming language is a group of strings which produce different forms of computer codes. 
    A programming language is a form of computer language that is used in implementing algorithms when programming computers.
    \section*{Some Common Programming Languages}
    \begin{itemize}
    	\item Basic
    	\item Fortran
    	\item C++
    	\item HTML
    	\item Java
    	\item JavaScript 
    	\item Python 
    	\item PHP
    	\item Ruby
    	\item CSS
    	\item Ada
    	\item COBOL etc.
    	\end{itemize}
    \section*{C++}
    C++began in the year 1979 by a man whose name was Bjarne Stroustrup when he worked on his Ph.d thesis.
    Before Stroustrup began to use C++ he worked with a programming language called simula which was a programming language used for simulations.
    Simula 67 language, which Stroustrup used, is credited with being the first to support the object-oriented programming paradigm. 
    This paradigm was discovered to be quite effective in software development but the Simula language was very slow to be practical.
    As a result, he began working on ”C with classes”, a new language that would combine the object-oriented paradigm with the capabilities of the C programming language. 
    Classes, inheritance, inlining, default function arguments, polymorphism, encapsulation, and strong type checking were all incorporated in the new language when it was released in 1983 and it was named C++.
    In October 1985, the first commercial edition of the C++ programming language    was launched. Similar programminglanguages: Python, Ruby, Java etc.
    \section*{Java}
    In the early 1990s, Sun Microsystems' James Gosling and colleagues created Java, an  object-oriented programming language.
    In June 1991, James Gosling began working on Java which was then called "Oak." 
    Gosling desired to make a virtual machine and a programming language which looked like C but was more  standardised and simpler than C/C++. 
    Java 1.0 was released in 1995 as the first public implementation.
    It was extremely secure and flexible, with network and file access constrained.
    The main  web browsers integrated fast into their regular settings after establishing a stable 'applet' arrangement.
    Following the release of Java 2,newerversions for large and small platforms (J2EE and J2ME)  quickly developed.
    Sun attempted to codify Java in 1997 by contacting the ISO/IEC JTC1 and later Ecma  International, but they immediately withdrew. 
    The Java Community Process continues to regulate Java as a de facto proprietary         standard. 
    Sun made some Java implementations free of charge thanks to cash produced by new visions like the Java Enterprise Framework. 
    The main distinction was that the compiler was not included in the JRE, which              distinguished it from its Software Development Kit (SDK).
    Similar Programming languages: C, JavaScript, Python, Scala etc.
    \section*{CSS}
    Hakon Wium Lie suggested CSS for the first time on the 10th of October, 1994. 
    During that period, he was working at CERN with Tim Berners-Lee, who is the father of HTML. 
    CERN stands for the European Organization for Nuclear Research. 
    Hakon Wium lie is regarded as the creator  of CSS.
    CSS stands for Cascading Style Sheet.
    CSS was introduced as a web style language in 1994 to address some of the issues with HTML 4.
    Other styling languages, such as Style Sheets for HTML and JSSS, were proposed at the time, but CSS won out.
    The ’cascading' method in CSS, which was discussed extensively in relation to distinct stylesallowed page managers to alter numerous sheets or pages at the same time.
    Different people were attempting to design other models at the time as well.
    The beauty of CSS, on the other hand, was that it was a relationship between the reader and the web author. 
    To put it another way, it was the belief that the document's style could not be established just by the writer or the reader; both parties' wants and desires had to be considered. 
    The design had to take into account the capabilities of the gadget that would be presenting the content.
    Similar Programming Languages: Java, Python, PHP, Ruby etc.
    \section*{Ruby}
    Yukihiro Matsumoto, also known as "Matz," invented Ruby in Japan in the   mid-1990s. 
    It was created with the thought that programming should be enjoyable for  programmers in mind. 
    It underlined the importance of software being understood first by humans and then by computers.
    Ruby's popularity in web application development continued to grow. 
    David Heinemeier Hansson created the Ruby on Rails framework, which      exposed many people to the joys of programming in Ruby.
    Ruby was first created in the year 1995, and ruby.95 was the first version of the language to be released.
    Yukihiro Matsumoto fully object-oriented programming language which could as well be used to write scripts. 
    Due to that, he formed the Ruby programming language.
    Similar Programming: Python, JavaScript, PHP, C++ etc.
    \section*{C}
    Dennis Ritchie, a brilliant computer scientist, invented the C programming             language in 1972. It was created by Bell Laboratories and AT and T, both of which are based in the United States. 
    Dennis Ritchie is widely regarded as the creator of the C programming                   language. 
    The B programming language predates the C programming language, and combines the features of ALGOL, BCPL, and B programming languages.
    To address the shortcomings of B programming language ,ALGOL and BCPL             languages the C  language was launched with a number of      new capabilities. 
    The C programming language was created primarily for the development of        Unix operating systems. 
    However, it is being used to create a variety of additional programmes and          applications.
    It's also called "ANSI C" since the American National Standards Institute (ANSI) established a commercial standard for the C language in 1989 in order to maintain it as a standard.
    Similar Programming languages: Go, Lua, Perl etc.
    \title*{Available IDEs For C++ Programming Language.}
    \begin{itemize}
    \item Visual studio code
    \item Code:: Blocks
    \item CLion
    \item Eclipse 
    \item CodeLite
    \item Apache NetBeans
    \item QT creator
    \item Dev C++
    \item C++ builder
    \item Xcode
    \item GNAT Programmers studio
    \item Kite 
    \item Sublime Text
    \item Brackets
    \item Atom etc.
     \end{itemize}
    \title*{Available IDEs For Java Programming Language.}
    \begin{itemize}
    	\item jEdit
    	\item NetBeans
    	\item jGRASP
    	\item Eclipse 
    	\item DrJava
    	\item JDeveloperAndroid Studio
    	\item Enide studio 2014
    	\item BlueJ
    	\item JSource
    	\item IntelliJ IDEA Community Edition
    	\item Kite etc.
    	\end{itemize}
    \section*{Available IDEs For CSS Programming Language.}
    \begin{itemize}
    \item Light Table
    \item PHPStorm
    \item Komodo Edit
    \item Atom by Github
    \item RJ TextEd
    \item Visual Studio Code
    \item Notepad++
    \item PyCharm 
    \item IntelliJ IDEA
    \item RubyMine
    \item NetBeans
    \item Sublime Text 3
    \item Brackets
    \item Webstorm 
\end{itemize}
    \section*{Available IDEs For Ruby Programming Language.}
       \begin{itemize}
       	\item RubyMine
       	\item Visual Studio Code
       	\item Atom
       	\item Aptana Studio
       	\item Eclipse
       	\item Komodo IDE
       	\item NetBeans
       	\item Selenium etc.
 \end{itemize}
    \section*{Available IDEs For C Programming Language.}
    \begin{itemize}
    	\item Eclipse 
    	\item Dev-C++
    	\item NetBeans
    	\item Visual Studio Code
    	\item CodeLite
    	\item GNAT Programming Studio
    	\item Code::Blocks
    	\item QT Creator
    	\item CodeWarrior
    	\item K Develop 
    	\item SlickEdit 
    	\item MinGWetc.
    \end{itemize}
	\section*{Application examples that exist or that can be developed using C++:}
	\begin{itemize}
		\item GUI based applications like Win Amp Media Player, Adobe Systems etc.GUI based applications like Win Amp Media Player, Adobe Systems etc.
		\item Operating Systems like Apple OS and Ipod have part of it  written in C++. The majority of Microsoft also has its software which is written C++.
		\item Browsers like Mozilla Firefox and Thunderbird are completely developed in C++.
		\item Games
		\item Google applications are also written in C++.
		\item Database Softwares like MySQL an Postgres are written using C++.
		\item Applications that need advanced computation and graphics etc.
	\end{itemize}
	\section*{Application examples that exist or that can be developed using Java:}
        \begin{itemize}
        	\item Java can be used to create web-based applicationslike Broadleaf use java In E-Commerce web applications..
        	\item It can also be used to create mobile applications like for example majority of ohones which have android OS  is devveloped by java. And mobile apps like Netflix, Google Calendar, Uber etc. uses java.
        	\item Enterprise Applications: companies like trivago, Spotify,google etc. Use java.
        	\item Cloud-based Applications.
         	\item Embedded Systems
        	\item Distributed Applications
        	\item Desktop GUI Applications
        	\item Web servers and Application servers
        	\item Software Tools
        	\item Scientific Applications
        	\item Gaming Applications: Because java supports Dalvik Virtual Machine (DVM) which is specifically built to perform on the android platform, android games use it as tgeir primary language.
        	\item Big Data Technologies etc.
        \end{itemize}
 \section*{Application examples that exist or that can be developed using CSS:}
   	   	\begin{itemize}
   	   		\item Animation and effects
   	   		\item Web-based Applications like Keyframes.app 
   	   		\item Website design
   	   		\item Social media  
   	   		\item CSSynth app which is a small app used to run animations.etc
   	   	\end{itemize}
    \section*{Application examples that exist or that can be developed using Ruby:}
   	  	\begin{itemize}
   	  		\item Hospitality services like AirBnB
   	  		\item Online music distribution like SoundCloud
   	  		\item Version control repositorylike GitHub
   	  		\item Project management systems like Basecamp
   	  		\item Online stores like Shopify
   	  		\item Live Video streaming like Hulu
   	  		\item Social Cataloging like GoodReads
   	  		\item Freelance marketplace like Fiverr etc.
        \end{itemize}	
    \section*{Application examples that exist or that can be developed using C:}
    \begin{itemize}
    	\item Systems like Microsoft Windows is powered by C.
    	\item Linux: is majorly Written in C.
    	\item Mac is powered in C.
    	\item Mobile phones IOS, Androids and windows are written in C. 
    	\item Databases like MySQL, Oracle Database, PostgreSQL are coded in C.
    	\item 3D movies are made in applications that are written in C.
    	\item Embedded Systems like alarm clock, Air bag control, child-proof locks etc. are programed in C.
    \end{itemize}
\end{document}
    
    